%% LaTeX-Beamer template for KIT design
%% by Erik Burger, Christian Hammer
%% title picture by Klaus Krogmann
%%
%% version 2.1
%%
%% mostly compatible to KIT corporate design v2.0
%% http://intranet.kit.edu/gestaltungsrichtlinien.php
%%
%% Problems, bugs and comments to
%% burger@kit.edu

\documentclass[18pt]{beamer}

%% SLIDE FORMAT

% use 'beamerthemekit' for standard 4:3 ratio
% for widescreen slides (16:9), use 'beamerthemekitwide'

\usepackage{templates/beamerthemekit}
%\usepackage{templates/beamerthemekitwide}

%% TITLE PICTURE

% if a custom picture is to be used on the title page, copy it into the 'logos'
% directory, in the line below, replace 'mypicture' with the 
% filename (without extension) and uncomment the following line
% (picture proportions: 63 : 20 for standard, 169 : 40 for wide
% *.eps format if you use latex+dvips+ps2pdf, 
% *.jpg/*.png/*.pdf if you use pdflatex)

%\titleimage{mypicture}

%% TITLE LOGO

% for a custom logo on the front page, copy your file into the 'logos'
% directory, insert the filename in the line below and uncomment it

% \titlelogo{kitlogo_de_rgb}

% (*.eps format if you use latex+dvips+ps2pdf,
% *.jpg/*.png/*.pdf if you use pdflatex)

%% TikZ INTEGRATION

% use these packages for PCM symbols and UML classes
% \usepackage{templates/tikzkit}
% \usepackage{templates/tikzuml}

% the presentation starts here

\title[Short title]{Full title\\With all details}
\subtitle{Subtitle}
\author{Firstname Lastname}

\institute{IT Security - Digitale Forensik}

% Bibliography
% Warning: natbib/apacite break hyperref support for citations

\usepackage[natbibapa]{apacite}
\bibhang1em

\begin{document}

% change the following line to "ngerman" for German style date and logos
\selectlanguage{ngerman}

%title page
\begin{frame}
\titlepage
\texttt{}\end{frame}

%table of contents
\begin{frame}{Outline/Gliederung}
\tableofcontents
\end{frame}

\section{Section 1}
\subsection{Subsection 1.1}
\begin{frame}{Example slide A}
\begin{itemize}
\item PCM
\item Citation: \citet{Gimpel2008} %\language
\pause
\item Bullet point 3
\item \dots
\end{itemize}
\end{frame}

\subsection{Subsection 1.2}
\begin{frame}{Example slide B}
\begin{block}{Block 1}
\begin{itemize}
\item Bullet point 1
\pause
\item Bullet point 2
\item \dots
\end{itemize}
\end{block}
\end{frame}

\section{Section 2}
\begin{frame}{Example slide C}
\begin{exampleblock}{Example 1}
\begin{itemize}
\item Bullet point 1
\pause
\item Bullet point 2
\item \dots
\end{itemize}
\end{exampleblock}
\end{frame}

\begin{frame}{Example slide D}
\begin{alertblock}{Alert 1}
\begin{itemize}
\item Bullet point 1
\pause
\item Bullet point 2
\item \dots
\end{itemize}
\end{alertblock}
\end{frame}

\appendix
\beginbackup

\begin{frame}[allowframebreaks]{References}
\bibliographystyle{apacite}
\bibliography{templates/example}
\end{frame}

\backupend

\end{document}
